\chapter{Исследовательский раздел}
Для проверки рабоспособности алгоритмов прогнозирования на неразмеченных данных, алгоритмы проверяюлись на размеченных данных.
Информация о матчах WTA(международная женская теннистная ассоциация) за период 2007-2019 года была взята с сайта data.world\cite{Book34}, информация о матчах ATP(мужская теннисная ассоциация) была взята с сайта opendatasoft.com\cite{Book35}.
Данные о ставочных коэффициентов были взяты с сайта oddsportal.com\cite{Book36}, теннисные интервью были взяты с сайта  ASAP
Sport’s\cite{Book37}, а так же из статьи Liye Fu\cite{Book38}.В некоторых случаях в качестве предматчевого интервью использовалось послематчевое интервью предыдущего матча. Все вопросы журналистов были вырезаны из интервью. Оставлены только ответы спорсменов на вопросы журналистов Так же для корректировки статистических данных использовался портал tennis-data\cite{Book39}. Набор данных разбивался на данные для обучения и данные для тестирования в пропорции 4 к 1.
На основе сбора данных с вышеприведенных источников был сформирован датасет следующих размеров: 533 матча WTA и 1202 матча ATP.
\section{Поиск оптимального набора стастистическхи свойств}
\textit{Тут будет выбор оптимального набора метрик при помощи метода хи-квадрат}
\section{Влияние тональности на точность прогнозов}
todo
\section{Сравнение алгоритмов прогнозирования теннисных матчей}
Сравним эффективность прогнозов полученного метода с другими аналогичными работами.
В некоторых работах не указана итоговая точность прогнозирования. Тогда точностью для данной работы считалась общее количество успешно предсказанных матчей, подёлённое на общее число проанализированных матчей\cite{Book40}. В некоторых работах не уточняется общее число проанализированных матчей.
\begin{table}[!h]
	
	\caption{\label{tab:issled1}Сравнение методов прогнозирования результатов теннисных матчей}
	
	\begin{center}
		
		\begin{tabular}{|l|l|l|l|l|}
			
			\hline
			
			Автор работы& Метод & Точность & Кол-во матчей& Турниры  \\
			
			\hline 
			
			McHale и & Модель& 66.90\% & - & ATP  \\
		    Morton, 2011  \cite{Book18} & Бредли-Терри& & &  \\
			\hline
			Scheibehenne и  & Агрегирование  & 70.06\% & 127 & ATP  \\
			Bröder,
			2007\cite{Book40} &рейтингов игроков&&& \\
			\hline
			Knottenbelt, Spanias,& Иерархическая & 69.38\% & 686& ATP \& WTA 	\\		Madurska, 2012\cite{Book41} & модель Маркова& && \\
			\hline
			Gu, Saaty,2019\cite{Book42} &Метод аналатич.& 85.10\%&94&ATP \& WTA\\
			  &сетей&&&\\
			\hline
			Метод данной &Нейронные сети& 68.29\%&347&ATP \& WTA\\
			работы &&&&\\
			\hline
		\end{tabular}
		
	\end{center}
	
\end{table}
Исходя из вышеприведенных данных можно заметить, что точность прогнозов уменьшается с увеличением количества аналазируемых данных.

Так же проведем сравнение алгоритмов прогнозирования для различных  ассоциаций турниров.
\begin{table}[!h]
	
	\caption{\label{tab:issled2}}{Сравнение методов прогнозирования результатов теннисных матчей с разбивкой по ассоциацях}
	
	\begin{center}
		
		\begin{tabular}{|l|l|l|l|l|}
		\hline
		
		Автор работы& Точн. ATP & Точн. WTA&ATP число матч. & WTA число матч. \\
		
		\hline 
		
		McHale и &66.90\%& - & - & -  \\
		Morton, 2011  \cite{Book18} & & & &  \\
		\hline
		Scheibehenne и  & 70.06\% & - & 127 & -  \\
		Bröder,
		2007\cite{Book40} &&&& \\
		\hline
		Knottenbelt, Spanias,& 70.30\% & 68.75\% & 270& 416	\\		Madurska, 2012\cite{Book41} & & && \\
		\hline
		Gu и  &87.4\%& 80.6\%&66&31\\
		 Saaty,2019\cite{Book42}&&&&\\
		\hline
		Метод данной &66.40\%& 72.64\%&241&106\\
		работы &&&&\\
		\hline
			
		\end{tabular}
		
	\end{center}
	
\end{table}

Исходя из количества истинно позитивных результатов и ложно позитивных результатов, можно рекомендовать использовать данный метод в задачах где акцент делается на нахождении истинно позитивных значений, пренебрегая некоторым количеством полученных ложно позитивных значений.
%%% Local Variables:
%%% mode: latex
%%% TeX-master: "rpz"
%%% End: