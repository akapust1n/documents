\Introduction

Задача поиска аномалий относится к одному из популярных способов машинного обучения - обучению без учителя. В настоящее время задачу поиска аномалий активно решают во многих областях жизнедеятельности:
\begin{enumerate} 
	\item Защита информации и безопастность
	\item Социальная сфера и медицина
	\item Банковская и финансовая отрасль
	\item Распознавание и обработка текста, изображений, речи
	\item Другие сферы деятельности(например, мониторинг неисправностей механизмов)
\end{enumerate}
Задачей поиска выбросов, как частный случай задачи поиска аномалий так же занимаются во вышеперичисленных отраслях. 

Количество данных в мире удваивается примерно каждые два года. Поэтому актуальной задачей является разработка новых методов и усовершенствования старых методов поиска выбросов.

В данной работе предлагается новый метод, позволяющий найти аномалии в выборках данных.




