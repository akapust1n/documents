\chapter{Аналитический раздел}
\label{cha:analysis}
\section{Цель и задачи работы}
Целью данной работы является создание программного комплекста для обнаружения выбросов временных рядов в собираемых данных.
Для достижения данной цели необходимо решить следующие задачи:
\begin{itemize}
	\item ПЕРЕПИСАТЬ ИЗ ПРЕЗЕНТАЦИИ
	\item пронализировать предметную область и существующие методы обнаружения выбросов
	\item разработать метод обнаружения выбросов
	\item smth
	\item создать ПО, реализующего  разработанный метод обнаружения выбросов
	\item провести вычислительный эксперименты с использованием разработанного метода
	
\end{itemize}
\section{Обнаружение аномалий}
В машинном обучении обнаружение  "ненормальных" экземпляров в наборах данных всегда представляло большой интерес. Этот процесс широко известен как обнаружение аномалий или обнаружение выбросов.  Вероятно, первое определение было дано Граббсом\cite{Book02} в 1969 году: "Относительное наблюдение или выброс - это элемент выборки, который, заметно отличается от других членов выборки, в которых он встречается ".
Хотя это определение по-прежнему актуально и сегодня, мотивация для обнаружения этих выбросов сейчас совсем другая. Тогда основная причина обнаружения заключалась в том, чтобы удалить выбросы из данных для обучения, поскольку алгоритмы распознавания  были весьма чувствительны к выбросам в данных. Эта процедура также называется очищением данных. После разработки более надежных классификаторов интерес к обнаружению аномалий значительно снизился. Однако в 2000 году произошел поворотный момент, когда исследователи стали больше интересоваться самими аномалиями, поскольку они часто связаны с особенно интересными событиями или подозрительными данными. С тех пор было разработано много новых алгоритмов, которые оцениваются в этой статье. В этом контексте определение Граббса также было расширено, так что сегодня аномалии, как известно, имеют две важные характеристики:
\begin{enumerate}
	\item Аномалия отличается от нормы по своим особенностям
	\item Аномалия редко встречается в наборе данных по сравнению с "нормальными" данными
\end{enumerate}
\subsection{Классификация методов обнаружений аномалий}
В отличие от хорошо известной  системы классификации, где учебные данные используются для обучения классификатора, а результаты измерений данных оцениваются впоследствии, возможно множество вариантов, когда речь идет об обнаружении аномалий. Метод обнаружения аномалий, которая будет использоваться, зависит от ярлыков, доступных в наборе данных, и мы можем выделить три основных типа:
\begin{enumerate}
\item Обучение с учителем. Доступны полностью размеченные данные для обучения и для тестов. Обычный классификатор может быть обучен один раз и применяться впоследствии. Это похоже на традиционное распозвание образов, за исключением того, что классы обычно сильно не сбалансированы. Поэтому не все алгоритмы классификации идеально подходят для этой задачи. Для многих применений аномалии не известны заранее или могут возникать спонтанно в качестве новинок на этапе тестирования.
\item Обучение с частичным привлечением учителя.Обучение использует учебные и тестовые наборы данных. Данные обучения состоят только из нормальных данных без каких-либо аномалий. Основная идея заключается в том, что модель нормального класса изучается, а аномалии могут быть обнаружены впоследствии, отклоняясь от этой модели. Эта идея также известна как «одноклассовая» классификация \cite{Book03}.
\item Обучение без учителя.Самая гибкий способ, который не требует каких-либо меток. Кроме того, нет различия между учебным и тестовым наборами данных. Идея заключается в том, что алгоритм обнаружения аномалии оценивает данные исключительно на основе внутренних свойств набора данных. Как правило, расстояния или плотности используются для оценки того, что является нормальным, а что является выбросом. В этой статье основное внимание уделяется этой неконтролируемой установке обнаружения аномалий.

\end{enumerate}











В анализе данных есть два основных направления, которорые занимаются поиском аномалий - это детектирование новизны и обнаружение выбросов. "Новый объект"- это так же объект, который отличается по своим свойствам от объектов  выборки. Однако, в отличие от выброса, он его ещё нет в самой выборке и задача анализа сводится к его обнаружению при появление. Например, если вы анализируете замеры уровня шума и отбрасываетете слишком высокие или слишком низкие значения, то вы боретесь с выбросам. А если Вы создаёте алгоритм, который для каждого нового замера оценивает, насколько он похож на прошлые, и выбрасывает аномальные — вы "боретесь с новизной"
\cite{Book01}.
Выбросы являются следствием:
\begin{enumerate}
	\item ошибок в данных
	\item неверно классифицированных объектов
	\item присутсвием объектов других выборок
	\item намеренным искажением данных
\end{enumerate}
\begin{figure}
	\centering
	\includegraphics[width=.5\textwidth]{img/1.png}
	\caption{Пример размеченного набора данных}
	\label{fig01}
\end{figure}

На рисунке \ref{fig01} можно увидеть желтые точки - выброс "слабом смысле". Они незначительно отклоняются от основных данных(зеленые точки). Красные же точки являются аномальными - выбросами "в сильном смысле", они значительно  отклоняются с от основывных данных. В данной работе будет изучаться вопрос находждения "сильных выбросов" и  критериев отличия сильного выброса от основных данных. В дальнейшем под словом "выброс" будет подразумеваться "сильный выброс",  а под  аномалией - в выброс(выброс является частным случаем аномалии).
Понятие аномалии зачастую интерпетируют по-разному в зависимости от характера данных. Обычно аномалией назыют некоторое отклонение от нормы. Это определение нуждается в формальном уточнении.

\section{Постановка задачи}

% Обратите внимание, что включается не ../dia/..., а inc/dia/...
% В Makefile есть соответствующее правило для inc/dia/*.pdf, которое
% берет исходные файлы из ../dia в этом случае.

%\begin{figure}
%  \centering
%  \includegraphics[width=\textwidth]{inc/dia/rpz-idef0}
%  \caption{Рисунок}
%  \label{fig:fig01}
%\end{figure}

%\begin{figure}
%  \centering
%  \includegraphics[height=0.85\textheight]{inc/img/leonardo}
%  \caption{Предполагаемый автопортрет Леонардо да Винчи}
 % \label{fig:leonardo}
%\end{figure}

%В \cite{Pup09} указано, что...

Кстати, про картинки. Во-первых, для фигур следует использовать \texttt{[ht]}. Если и после этого картинки вставляются <<не по ГОСТ>>, т.е. слишком далеко от места ссылки, "--- значит у вас в РПЗ \textbf{слишком мало текста}! Хотя и ужасный параметр \texttt{!ht} у окружения \texttt{figure} тоже никто не отменял, только при его использовании документ получается страшный, как в ворде, поэтому просьба так не делать по возможности.

\section{Существующие подходы к созданию всячины}

Известны следующие подходы...

\begin{enumerate}
\item Перечисление с номерами.
\item Номера первого уровня. Да, ГОСТ требует именно так "--- сначала буквы, на втором уровне "--- цифры.
Чуть ниже будет вариант <<нормальной>> нумерации и советы по её изменению.
Да, мне так нравится: на первом уровне выравнивание элементов как у обычных абзацев. Проверим теперь вложенные списки.
\begin{enumerate}
\item Номера второго уровня.
\item Номера второго уровня. Проверяем на длииииной-предлиииииииииинной строке, что получается.... Сойдёт.
\end{enumerate}
\item По мнению Лукьяненко, человеческий мозг старается подвести любую проблему к выбору
  из трех вариантов.
\item Четвёртый (и последний) элемент списка.
\end{enumerate}

Теперь мы покажем, как изменить нумерацию на «нормальную», если вам этого захочется. Пара команд в начале документа поможет нам.

\renewcommand{\labelenumi}{\arabic{enumi})}
\renewcommand{\labelenumii}{\asbuk{enumii})}

\begin{enumerate}
\item Изменим нумерацию на более привычную...
\item ... нарушим этим гост.
\begin{enumerate}
\item Но, пожалуй, так лучше.
\end{enumerate}
\end{enumerate}

В заключение покажем произвольные маркеры в списках. Для них нужен пакет \textbf{enumerate}.
\begin{enumerate}[1.]
\item Маркер с арабской цифрой и с точкой.
\item Маркер с арабской цифрой и с точкой.
\begin{enumerate}[I.]
\item Римская цифра с точкой.
\item Римская цифра с точкой.
\end{enumerate}
\end{enumerate}

В отчётах могут быть и таблицы "--- см. табл.~\ref{tab:tabular} и~\ref{tab:longtable}.
Небольшая таблица делается при помощи \Code{tabular} внутри \Code{table} (последний
полностью аналогичен \Code{figure}, но добавляет другую подпись).

\begin{table}[ht]
  \caption{Пример короткой таблицы с коротким названием}
  \begin{tabular}{|r|c|c|c|l|}
  \hline
  Тело      & $F$ & $V$  & $E$ & $F+V-E-2$ \\
  \hline
  Тетраэдр  & 4   & 4    & 6   & 0         \\
  Куб       & 6   & 8    & 12  & 0         \\
  Октаэдр   & 8   & 6    & 12  & 0         \\
  Додекаэдр & 20  & 12   & 30  & 0         \\
  Икосаэдр  & 12  & 20   & 30  & 0         \\
  \hline
  Эйлер     & 666 & 9000 & 42  & $+\infty$ \\
  \hline
  \end{tabular}
  \label{tab:tabular}
\end{table}

Для больших таблиц следует использовать пакет \Code{longtable}, позволяющий создавать
таблицы на несколько страниц по ГОСТ.

Для того, чтобы длинный текст разбивался на много строк в пределах одной ячейки, надо в
качестве ее формата задавать \texttt{p} и указывать явно ширину: в мм/дюймах
(\texttt{110mm}), относительно ширины страницы (\texttt{0.22\textbackslash textwidth})
и~т.п.

Можно также использовать уменьшенный шрифт "--- но, пожалуйста, тогда уж во \textbf{всей}
таблице сразу.

\begin{center}
  \begin{longtable}{|p{0.40\textwidth}|c|p{0.30\textwidth}|}
    \caption{Пример длинной таблицы с длинным названием на много длинных-длинных строк}
    \label{tab:longtable}
    \\ \hline
    Вид шума & Громкость, дБ & Комментарий \\
    \hline \endfirsthead
    \subcaption{Продолжение таблицы~\ref{tab:longtable}}
    \\ \hline \endhead
    \hline \subcaption{Продолжение на след. стр.}
    \endfoot
    \hline \endlastfoot
    Порог слышимости             & 0     &                                                \\
    \hline
    Шепот в тихой библиотеке     & 30    &                                                \\
    Обычный разговор             & 60-70 &                                                \\
    Звонок телефона              & 80    & \small{Конечно, это было до эпохи мобильников} \\
    Уличный шум                  & 85    & \small{(внутри машины)}                        \\
    Гудок поезда                 & 90    &                                                \\
    Шум электрички               & 95    &                                                \\
    \hline
    Порог здоровой нормы         & 90-95 & \small{Длительное пребывание на более
    громком шуме может привести к ухудшению слуха}                                        \\
    \hline
    Мотоцикл                     & 100   &                                                \\
    Power Mower                  & 107   & \small{(модель бензокосилки)}                  \\
    Бензопила                    & 110   & \small{(Doom в целом вреден для здоровья)}     \\
    Рок-концерт                  & 115   &                                                \\
    \hline
    Порог боли                   & 125   & \small{feel the pain}                          \\
    \hline
    Клепальный молоток           & 125   & \small{(автор сам не знает, что это)}          \\
    \hline
    Порог опасности              & 140   & \small{Даже кратковременное пребывание на
    шуме большего уровня может привести к необратимым последствиям}                       \\
    \hline
    Реактивный двигатель         & 140   &                                                \\
                                 & 180   & \small{Необратимое полное повреждение
                                 слуховых органов}                                        \\
    Самый громкий возможный звук & 194   & \small{Интересно, почему?..}                   \\
  \end{longtable}
\end{center}

%%% Local Variables:
%%% mode: latex
%%% TeX-master: "rpz"
%%% End:
