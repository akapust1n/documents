% Также можно использовать \Referat, как в оригинале
%\begin{abstract}
%	Титульный лист. Эта страница нужна мне, чтобы не сбивалась нумерация страниц
%	\cite{Dh}
%	\cite{Bayer}
%	\cite{Habr1}
%	\cite{Noise_func}
%	\cite{Ulich}

%Это пример каркаса расчётно-пояснительной записки, желательный к использованию в РПЗ проекта по курсу РСОИ.

%Данный опус, как и более новые версии этого документа, можно взять по адресу (\url{https://github.com/rominf/latex-g7-32}).

%Текст в документе носит совершенно абстрактный характер.
%\end{abstract}
% НАЧАЛО ТИТУЛЬНОГО ЛИСТА
\begin{center}
	\hfill \break
	\textit{
		\normalsize{Государственное образовательное учреждение высшего профессионального образования}}\\ 
	
	\textit{
		\normalsize  {\bf  «Московский государственный технический университет}\\ 
		\normalsize  {\bf имени Н. Э. Баумана»}\\
		\normalsize  {\bf (МГТУ им. Н.Э. Баумана)}\\
	}
	\noindent\rule{\textwidth}{2pt}
	\hfill \break
	\noindent
	\makebox[0pt][l]{ФАКУЛЬТЕТ}%
	\makebox[\textwidth][c]{«Информатика и системы управления»}%
	\\
	\noindent
	\makebox[0pt][l]{КАФЕДРА}%
	\makebox[\textwidth][r]{«Программное обеспечение ЭВМ и информационные технологии»}%
	\\
	\hfill\break
	\hfill \break
	\hfill \break
	\hfill \break
	\normalsize{\bf Р А С Ч Ё Т Н О - П О Я С Н И Т Е Л Ь Н А Я\space\space З А П И С К А}\\
	\normalsize{\bf к курсовой работе на тему:}\\
	\hfill \break
	\large{Метод сохранения качества изображений при понижении его разрядности}\\
	\hfill \break
	\hfill \break
	\hfill \break
	\hfill \break
	\hfill \break	
	\normalsize {
		\noindent
		\makebox[0pt][l]{Студент}%
		\makebox[\textwidth][c]{Капустин А.И.}%
		\makebox[0pt][r]{{$\underset{\text{(Подипсь, дата)}}{\underline{\hspace{8cm}}}$ \space И.О. Фамилия}}
	}\\
	\hfill \break	
	\normalsize {
		\noindent
		\makebox[0pt][l]{Руководитель курсового проекта}%
		\makebox[\textwidth][c]{        Оленев А.А.}%
		\makebox[0pt][r]{{$\underset{\text{(Подпись, дата)}}{\underline{\hspace{6cm}}}$ \space И.О. Фамилия}}
	}
	\hfill \break
	\hfill \break
	\hfill \break
	\hfill \break
\end{center}
\hfill \break
\hfill \break
\begin{center} Москва 2016 \end{center}

\thispagestyle{empty} % 
% КОНЕЦ ТИТУЛЬНОГО ЛИСТА


%%% Local Variables: 
%%% mode: latex
%%% TeX-master: "rpz"
%%% End: 