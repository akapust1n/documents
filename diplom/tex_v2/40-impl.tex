\chapter{Технологический раздел}
НЕ ГОТОВО
\section{Клиентская часть}

\subsection{Библиотека KUserFeedback}

В качестве вспомогательной библиотеки для сбора стастики используется библиотека KUserFeedback компании KDAB. Эта библиотека включает в себя С++ Qt клиентскую часть, а так же сервер, написанный на PHP\cite{book3} Нам не требуется их сервер и значительная часть функций, мы будем использовать только часть собирающую телеметрию. KUserFeedBack позволяет  собрать библиотеку по частям и линковать к нашему приложению только необхоимые модули. В программе используется модуль Core. Модуль Core содержит абстракный класс источника данных, от которого можно наследовать различные источники данных. Ниже представлено описание этого класса. В дальнейшем мы создадим классы, которые наследуются от этого абстрактого класса. В этих классах будут переопределеные все чисто виртуальные функции. Ключевую роль будет играть переопределение функции \textit{data()} - именно это переопределение несет смысловую нагрузку класса.
\begin{lstlisting}[language=c++,,escapeinside={(@}{@)},caption={Описание базового класса библиотеки}] 
class KUSERFEEDBACKCORE_EXPORT AbstractDataSource
{
public:
virtual ~AbstractDataSource();
QString name() const;
virtual QString description() const = 0;
virtual QVariant data() = 0;
virtual void load(QSettings *settings);  
virtual void store(QSettings *settings);
virtual void reset(QSettings *settings);
Provider::TelemetryMode telemetryMode() const;
void setTelemetryMode(Provider::TelemetryMode mode);

protected:

explicit AbstractDataSource(const QString &name, Provider::TelemetryMode mode = Provider::DetailedUsageStatistics, AbstractDataSourcePrivate *dd = nullptr);
void setName(const QString &name);
class AbstractDataSourcePrivate* const d_ptr;
};
}


\end{lstlisting}
