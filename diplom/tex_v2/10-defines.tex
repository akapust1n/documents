\Defines % Необходимые определения. Вряд ли понадобться
\begin{description}
	
	%\textbf{test} 
	\item[Выборка/выборка данных]  конечный набор прецедентов (объектов, случаев, событий, испытуемых, образцов, и т.п.), некоторым способом выбранных из множества всех возможных прецедентов, называемого генеральной совокупностью\cite{def01}.
	\item[Метка(ярлык)]  - порция данных, идентифицирующая набор данных, описывающая его определенные свойства и обычно хранимая в том же пространстве памяти, что и набор данных\cite{def02}.
	
	\item[Задача классификации]{ формализованная задача, в которой имеется множество объектов (ситуаций), разделённых некоторым образом на классы. Задано конечное множество объектов, для которых известно, к каким классам они относятся. Это множество называется выборкой. Классовая принадлежность остальных объектов неизвестна. Требуется построить алгоритм, способный классифицировать  произвольный объект из исходного множества.\cite{def03}}
	\item[Классификатор] {алгоритм, решающий задачу классификации.}
	
	\item[Теория распознавания образа] раздел информатики и смежных дисциплин, развивающий основы и методы классификации и идентификации предметов, явлений, процессов, сигналов, ситуаций и т.п. объектов, которые характеризуются конечным набором некоторых свойств и признаков.

	\item[Датасет]  набор данных.\cite{def03}
	
	\item[Схождением] называется такое состояние популяции, когда все строки популяции почти одинаковы и находятся в области некоторого экстремума. \cite{def05}
	
\end{description}

%%% Local Variables:
%%% mode: latex
%%% TeX-master: "rpz"
%%% End:
