\Defines % Необходимые определения. Вряд ли понадобться
\begin{description}
	
	%\textbf{test} 
	\item[Выборка/выборка данных]  конечный набор прецедентов (объектов, случаев, событий, испытуемых, образцов, и т.п.), некоторым способом выбранных из множества всех возможных прецедентов, называемого генеральной совокупностью\cite{def01}.
	\item Метка(ярлык)  - порция данных, идентифицирующая набор данных, описывающая его определенные свойства и обычно хранимая в том же пространстве памяти, что и набор данных\cite{def02}.
	
	классификатор
	\item[Теория распознавания образа] раздел информатики и смежных дисциплин, развивающий основы и методы классификации и идентификации предметов, явлений, процессов, сигналов, ситуаций и т.п. объектов, которые характеризуются конечным набором некоторых свойств и признаков.
	ddos-атака?
	Датасет - набор данных\cite{def03}
	
\end{description}

%%% Local Variables:
%%% mode: latex
%%% TeX-master: "rpz"
%%% End:
