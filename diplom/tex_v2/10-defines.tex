\Defines % Необходимые определения. Вряд ли понадобться
\begin{description}
	
	%\textbf{test} 
	\item[Выборка/выборка данных]  конечный набор прецедентов (объектов, случаев, событий, испытуемых, образцов, и т.п.), некоторым способом выбранных из множества всех возможных прецедентов, называемого генеральной совокупностью\cite{def01}.
	\item[Метка(ярлык)]  порция данных, идентифицирующая набор данных, описывающая его определенные свойства и обычно хранимая в том же пространстве памяти, что и набор данных\cite{def02}.
	
	\item[Задача классификации]{ формализованная задача, в которой имеется множество объектов (ситуаций), разделённых некоторым образом на классы. Задано конечное множество объектов, для которых известно, к каким классам они относятся. Это множество называется выборкой. Классовая принадлежность остальных объектов неизвестна. Требуется построить алгоритм, способный классифицировать  произвольный объект из исходного множества\cite{def03}}.
	\item[Классификатор] {алгоритм, решающий задачу классификации.}
	
	\item[Теория распознавания образа] раздел информатики и смежных дисциплин, развивающий основы и методы классификации и идентификации предметов, явлений, процессов, сигналов, ситуаций и т.п. объектов, которые характеризуются конечным набором некоторых свойств и признаков.

	\item[Датасет]  набор данных\cite{def07}.
	\item[Популяция]  множество возможных значений параметра алгоритма(в контексте генетических алгоритмов)\cite{Book17}.
	\item[Схождением] называется такое состояние популяции, когда все строки популяции почти одинаковы и находятся в области некоторого экстремума \cite{def05}.
	\item[JSON] текстовый формат обмена данными, основанный на JavaScript\cite{def08}.
	\item[Krita]  растровый графический редактор с открытым кодом, программное обеспечение
	\item[Функция правдоподобия] в математической статистике - это совместное распределение выборки из параметрического распределения, рассматриваемое как функция параметра. При этом используется совместная функция плотности (в случае выборки из непрерывного распределения) либо совместная вероятность (в случае выборки из дискретного распределения), вычисленные для данных выборочных значений\cite{Book21}
	\item[Размеченные данные]  набор данных и классов данных, размеченный особым образом, позволяющий однозначно определеить принадлежность каждого элемента данных к определенному классу. 
	
\end{description}

%%% Local Variables:
%%% mode: latex
%%% TeX-master: "rpz"
%%% End:
