\chapter{Исследовательский раздел}
Ошибки первого и второго рода - ключевые понятие математической статистики, помогающие проверять статистические гипотезы. Они так же используются в областях, где речь идет о принятие бинарного решения(да/нет). Задача поиска аномалий сводится к задаче бинарной классикации, где вопрос, помогающий осущесвить классификацию, звучит следующий образом: "Является ли точка аномалий?". 

Возьмум гипотезу H0: "Точка является аномалий". Тогда можно составить следующую таблицу классификации ошибок:
\begin{table}[!h]
	
	\caption{\label{tab:truefalse}Описание ошибок первого и второго рода}
	
	\begin{center}
		
		\begin{tabular}{|l|l|l|}
			
			\hline
			
			H0 & Верная & Ложная \\
			
			\hline \hline
			
			 Отклоняется & Ошибка первого рода & Решение верное \\
			
			\hline 
	       Не отклоняется & Решение верное  & Ошибка второго рода\\
						\hline 
			
		\end{tabular}
		
	\end{center}
	
\end{table}

В машнном обучении используют другую терминологию для описания ошибок первого и второго рода. Это позволяет выделить 4 класса результа принятия решения(а не 3 как принятно в математической статистике).
\begin{table}[!h]
	
	\caption{\label{tab:truefalse2}Описание ошибок первого и второго рода в терминогологии машинного обучения}
	
	\begin{center}
		
		\begin{tabular}{|l|l|l|}
			
			\hline
			
			H0 & Верная & Ложная \\
			
			\hline \hline
			
			Отклоняется & Ложное негативное & Истинное негативное \\
			
			\hline 
			Не отклоняется & Истинное позитивное  & Ложное позитивное\\
			\hline 
			
		\end{tabular}
		
	\end{center}
	
\end{table}
На основании значений этих метрик составялются метрики полноты, точности и графика под площадью РОК-кривой(описан выше)
\section{Полнота и точность. Площадь под РОК-кривой}
Точность(precision) - это доля объектов, названных классификатором положительными и при этом действительно являющимися положительными
	\begin{equation}
precision= \frac{TP}{TP+FP}
	\end{equation}
TP- количество истинно позитивных, FP - количестно ложно позитивных.
	\newline
	\begin{equation}
	recall= \frac{TP}{TP+FN}
	\end{equation}
	Полнота(recall) показывает сколько объектов положительного класса из всех объектов положительного класса нашел алгоритм.
	Мера точности не позволяет нам записывать все объекты в один класс, так как в этом случае мы получаем рост уровня ложно позитивных. Полнота демонстрирует способность алгоритма обнаруживать данный класс вообще, а точность— способность отличать этот класс от других классов. Полнота и точность могут применяться в условиях когда классы несбалансированы\cite{def06}. Поэтому они могут применять в задаче поиска аномалий, которая подрузамевает несбалансированность классов.
	Так же существует метрика называемая F1-показатель. Она является средним гармоническим величин полноты и точности.
	Метрика площади под РОК-кривой описывается выше, здесь же приведем формулу её расчёта.
	\begin{equation}
	TPR=\frac{TP}{TP+FN}
	\end{equation}
		\begin{equation}
	FPR=\frac{FP}{TN+FP}
		\end{equation}
		

		\begin{equation}
	ROC=\frac{1 + TPR - FPR}{2} 
		\end{equation}
	
\section{Сравнение  алгоритмов поиска аномалий}
Для проверки работоспособности алгоритмов поиска аномалий на неразмеченных данных, эти алгоритмы проверялись на размеченных данных.
Для этого работа алгоритма поиска аномалий была протестирована на двух наборах данных:
\begin{table}[!h]
	
	\caption{\label{tab:issled1}Характеристики датасетов, метрики полноты и точности}
	
	\begin{center}
		
		\begin{tabular}{|l|l|l|l|l|l|}
			
			\hline
			
			Набор данных& Кол-во элем. & Кол-во атриб. &  Полнота & Точн.& Кол-во аном.  \\
			
			\hline 
			
			WBC& 453 & 9 & 0.99&0.94 & 10  \\
			
			\hline
			KDDCUP99 & 60853 & 41 & 0.93&0.06 & 246  \\
			\hline
		
			
		\end{tabular}
		
	\end{center}
	
\end{table}
\begin{table}[h]
	
	\caption{\label{tab:issled2}Сравнение  алгоритмов поиска аномалий}
	
	\begin{center}
		
		\begin{tabular}{|l|l|l|l|l|}
			
			\hline
			
			Алгоритм & AUC ROC WBC & F1 WBC &  AUC ROC KDD & F1 KDD \\
			
			\hline 
			
			LoOp& 0.98 & 0.72 & 0.68& 0.05  \\
			
			\hline
			ODIN & 0.62 & 0.80 & 0.80& 0.06  \\
			
			\hline 
			KDEOS & 0.25	 & 0.64 & 0.61& 0.05  \\
			
			\hline 
			LDOF & 0.64	 & 0.96 & 0.88&0.07  \\
			\hline 
			INFLO & 0.99	 & 0.9 & 0.98&0.29  \\
			\hline   
			Разр. алгоритм & 0.92	 & 0.97 & 0.93 & 0.06  \\
			
			\hline  
			
		\end{tabular}
		
	\end{center}
	
\end{table}
Проведем сравнения с другими алгоритмами поиска аномалий.
Как можно увидеть из результатов  метрик AUC ROC и F1, алгоритмы по-разному классифицируют  разные наборы данных.Например, алгоритм LoOP показывает высокий AUC ROC на первом наборе данных, но на втором наборе данных его показали значительно снижаются. В свою очередь, алгоритм ODIN показывает низкие результаты, по сравнению с остальными алгоритмами, на первом наборе данных, но на втором наборе данных его AUC ROC высок. Разработанной алгоритм показывает средние значения AUC ROC, но высокие значения показателя F1, что позволяет утверждать, что этот алгоритм жизнеспособен и возможно его применение на определенных наборах данных. 
\section{Рекомендации к использованию метода}
\begin{table}[h]
	
	\caption{\label{tab:issled2}Количество истинно/ложно позитивно/негативно классифицировавшихся}
	
	\begin{center}
		
		\begin{tabular}{|l|l|l|l|l|}
			
			\hline
			
			Набор данных & ИП & ЛП &  ИН & ЛН \\
			
			\hline 
			
			WBC & 10 & 18 & 425 & 0  \\
			\hline 
				
			KDDCUP99& 230 & 3603 & 57004& 16  \\	 
			
			\hline  
			
		\end{tabular}
		
	\end{center}
	
\end{table}
Исходя из количества истинно позитивных результатов и ложно позитивных результатов, можно рекомендовать использовать данный метод в задачах где акцент делается на нахождении истинно позитивных значений, пренебрегая некоторым количеством полученных ложно позитивных значений.
%%% Local Variables:
%%% mode: latex
%%% TeX-master: "rpz"
%%% End:
