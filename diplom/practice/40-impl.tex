\chapter{Технологические средства}

\section{Выбор средств разработки}
\subsection{Язык программирования и средства разработки}
Для реализации данных алгоритмов был выбран язык  С++. Данный язык был обоснован следующими причинами:
Причины:
\begin{enumerate}
	\item Компилируемый язык со статической типизацией. 
	\item Сочетание высокоуровневых и низкоуровневых средств.
	\item Реализация ООП.
	\item Наличие удобной стандартной библиотеки шаблонов
\end{enumerate}
В качестве средств разработки была выбрана Qt Creator,
поддерживающая все возможности языка C++ и имеющий инструментарий
для создания как консольных приложений, так и приложений с графическим
интерфейсом.
\subsection{Программа для сбора данных}
В качестве языка написания плагина к графическому редактору был выбран язык С++ в силу того, что сам редактор поддерживает только плагины на этом языка. 
Для написания бекенд-сервера был выбран язык Golang вместе с библиотекой  mgo для доступа к базе данных . Плюсы этого языка:
\begin{enumerate}

\item Скорость разработки	 	

\item Производительность

\item Удобная реализация легковесных потоков 


\end{enumerate}
