\chapter{Технологический раздел}
\section{Выбор  языка программирования}
Для реализации загружаемого модуля был выбран язык С(приложение \ref{cha:appendix2}). 
Операционная система Linux позволяет писать загружаемые модули ядра на Rust и на C. 
Rust непопулярен и ещё только развивается и не обладает достаточным количеством документации.
Для реализации агрегатора и клиента был выбран язык С++.
\begin{enumerate}
	 \item Компилируемый язык со статической типизацией. 
	 \item Сочетание высокоуровневых и низкоуровневых средств.
	 \item Реализация ООП.
	 \item Наличие удобной стандартной библиотеки шаблонов
	 \end{enumerate}
\section{Выбор вспомогательных библиотек}
Для реализации программы была выбрана библиотека Qt.
\begin{enumerate}
	\item Широкие возможности работы с изображениями, в том числе и попиксельно
	\item Наличии более функциональных аналогов стандартной библиотеки шаблонов в том числе для разнообразных структур данных
\end{enumerate}




%%% Local Variables:
%%% mode: latex
%%% TeX-master: "rpz"
%%% End:
