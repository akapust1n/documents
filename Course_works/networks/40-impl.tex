\chapter{Технологический раздел}
\section{Плагин}
Для написания плагина к графическому редактору был выбран язык С++  в связи с тем, что сам графический редактор написан на этом языке. Преимущества С++:
\begin{enumerate}
	 \item Компилируемый язык со статической типизацией. 
	 \item Сочетание высокоуровневых и низкоуровневых средств.
	 \item Реализация ООП.
	 \item Наличие удобной стандартной библиотеки шаблонов
	 \end{enumerate}
	 
\section{Backend-сервер}
	 Для написания бекенд-сервера был выбран язык Golang вместе с библиотекой  mgo для доступа к базе данных . Плюсы этого языка:
	 \begin{enumerate}
	 	\item Скорость разработки	 	
	 	\item Производительность
	 	\item Удобная реализация легковесных потоков \cite{book5}

	 \end{enumerate}

\section{Фронтенд-сервер}
В качестве языка разработки фронтенд-сервера был выбран Python.
Хорошо известно, что Python является одним из самых используемых языков программирования благодаря простоте в изучении, дизайну и гибкости, что делает его практически совершенным языком программирования\cite{book4} Существует ряд причин, по которым его можно называть такими громкими словами.
	 \begin{enumerate}
	\item Простота в изучении
	\item Чистота и читаемость
	\item Разносторонность
	\item Быстрота написания
	\item  Цельный дизайн
	 \end{enumerate}
	 В качестве вспомогательного фрейворка был выбран Django. Плюсы Django:
	 	 \begin{enumerate}
	\item Быстрота: Django был разработан, чтобы помочь разработчикам создать приложение настолько быстро, на сколько это возможно. Это включает в себя формирование идеи, разработку и выпуск проекта, где Django экономит время и ресурсы на каждом из этих этапов. 
	 	\item Полная комплектация: Django работает с десятками дополнительных функций, которые заметно помогают с аутентификацией пользователя, картами сайта, администрированием содержимого, RSS и многим другим. Данные аспекты помогают осуществить каждый этап веб разработки.
		\item  Безопасность: Работая в Django, вы получаете защиту от ошибок, связанных с безопасностью и ставящих под угрозу проект. Я имею ввиду такие распространенные ошибки, как инъекции SQL, кросс-сайт подлоги, clickjacking и кросс-сайтовый скриптинг. Для эффективного использования логинов и паролей, система пользовательской аутентификации является ключом.
	 	\item  Масштабируемость: фреймворк Django наилучшим образом подходит для работы с самыми высокими трафиками. Следовательно, логично, что великое множество загруженных сайтов используют Django для удовлетворения требований, связанных с трафиком.

	 	 \end{enumerate}
\section{База данных}
В связи с возможной изменчивостью данных, приходящих с клиента, было решено использовать MongoDB.
Плюсы MongoDB:
	 \begin{enumerate}
	 	\item Бесструктурность
	 	\item Возможность масштабировать сервера "из коробки"
	 	\item Удобный NoSql синтаксис запросов
	 	\end {enumerate}

%%% Local Variables:
%%% mode: latex
%%% TeX-master: "rpz"
%%% End:
