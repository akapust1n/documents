\chapter{Аналитический раздел}
\label{cha:analysis}
%
% % В начале раздела  можно напомнить его цель
%
\section{Телеметрия}
Телеметрия - это автоматическая запись и передача данных из удаленных или недоступных источников в ИТ-систему в другом месте для мониторинга и анализа. Данные телеметрии могут быть переданы с использованием радио, инфракрасного, ультразвукового, GSM, спутника или кабеля, в зависимости от приложения (телеметрия используется не только в разработке программного обеспечения, но также в метеорологии, разведке, медицине и других областях).
В данном случае телеметрия - это слежение за действиями пользователя в приложении, установленном на персональном компьютере пользователя.
\subsection{Как работает телеметрия}
В общем случае телеметрия работает через датчики в удаленном источнике, который измеряет физические (например, осадки, давление или температуру) или электрические (такие как данные о токе или напряжении) или же через специальные программные средства).Совокупность данных, полученных из разных источников формируют поток данных, который передается по беспроводной среде, проводной или комбинации обоих.
В контексте разработки программного обеспечения понятие телеметрии часто путается с журналированием. Но журнал - это инструмент, используемый в процессе разработки для диагностики ошибок и потоков кода, и он ориентирован на внутреннюю структуру веб-сайта, приложения или другого проекта разработки. Однако, как только проект будет выпущен, телеметрия - это то, что вы ищете для автоматического сбора данных из реального мира. Телеметрия - это то, что позволяет собирать все необработанные данные, которые становятся ценными, эффективными для аналитиков.
В удаленном приемнике поток дезагрегирован, а исходные данные отображаются или обрабатываются на основе пользовательских спецификаций.

В мире разработки программного обеспечения телеметрия может дать представление о том, какие функции используют конечные пользователи, обнаружение ошибок и проблем, а также улучшенная видимость производительности без необходимости запрашивать обратную связь непосредственно от пользователей.
\subsection{Преимущества телеметрии}
Основным преимуществом телеметрии является способность конечного пользователя контролировать состояние объекта или окружающей среды, физически далеко от него. После того, как вы отправили продукт, вы не сможете физически присутствовать, просматривая плечи тысяч (или миллионов) пользователей, когда они взаимодействуют с вашим продуктом, чтобы узнать, что работает, что легко и что громоздко. Благодаря телеметрии эти идеи могут быть доставлены непосредственно к вам на компьютер, чтобы вы могли анализировать и действовать.

Поскольку телеметрия дает представление о том, насколько хорошо ваш продукт работает для ваших конечных пользователей, когда  они его используют, это невероятно ценный инструмент для постоянного мониторинга и управления производительностью. Кроме того, вы можете использовать собранные вами данные из версии 1.0 для улучшения и обновления приоритетов для выпуска версии 2.0.
Телеметрия позволяет вам отвечать на такие вопросы, как:
\begin{itemize}
	\item 	Ваши клиенты используют функции, которые вы ожидаете? Как они взаимодействуют с вашим продуктом?
	\item Как часто пользователи взаимодействуют с вашим приложением и на какой срок?
     \item Какие параметры настройки для пользователей выбираются чаще всего?
     \item Предпочитают ли они определенные типы дисплеев, методы ввода, ориентацию экрана или другие конфигурации устройства?
     \item  Что происходит при сбоях? Каков контекст, связанный с сбоями?
\end{itemize}
Очевидно, что ответы на эти и многие другие вопросы, на которые можно ответить с помощью телеметрии, неоценимы для процесса разработки, позволяя вам постоянно совершенствовать и внедрять новые функции, которые могут показаться конечным пользователям полученными из их мыслей(а вы, конечно, внедрили их благодаря телеметрии).
\subsection{Проблемы телеметрии}
Телеметрия - это, безусловно, фантастическая технология, но и она имеет свои недостатки. Самая важная проблема - и часто встречающаяся проблема - связана не с самой телеметрией, а с вашими конечными пользователями и их готовностью разрешить собирать данные. Некоторые видят телеметрию как  шпионаж . Короче говоря, некоторые пользователи сразу же отключают телметрию, когда замечают её, то есть любые данные, полученные от использования вами вашего продукта, не будут собираться или сообщаться.

Это означает, что опыт этих пользователей не будет учитываться при планировании будущей дорожной карты, исправлении ошибок или решении других проблем в вашем приложении. Пользователи, которые склонны отказываться от сбора статистики, могут негативно повлиять на ваше планы разработки. Другие пользователи, с другой стороны, не обращают внимания на телеметрию, происходящую за кулисами, или просто игнорируют ее, если они это делают.

Это проблема не имеет четкого решения - и это не отрицает общую силу телеметрии для процесса разработки, но ее следует учитывать при анализе ваших данных.
\section{Графический редактор Krita}
\textit{Krita} — бесплатный растровый графический редактор с открытым кодом, программное обеспечение, входящее в состав KDE. Ранее распространялось как часть офисного пакета Calligra Suite, но впоследствии отделилось от проекта и стало развиваться самостоятельно. Разрабатывается преимущественно для художников и фотографов, распространяется на условиях GNU GPL.

Начало разработки Криты было положены Матасом Этрикхом в 1998 года на конференции Linux 1998. Маттиас хотел показать возможно создать Qt GUI для уже существующего приложение и в качествея демо-приложения он выбрал Gimp. Его патч никогда не публиковался, но он вызвал споры в сообществе Gimp.

Не имея возможности работать вместе, люди в рамках проекта KDE решили запустить собственное приложение для редактора изображений. Основное внимание было уделено приложению, которое было частью KOffice, названного KImage, Майклом Кохом. Переименованный в KImageShop, это было началом Криты.

31 мая 1999 года проект KImageShop официально стартовал с почтовый рассылки Маттиаса Элтера. Основная идея тогда заключалась в том, чтобы сделать KImageShop оболочкой графического интерфейса вокруг ImageMagick. Они планировали, что это будет приложение  на основе corba,  с плагинами,  совместимое с плагинами GIMP.

Название KImageShop не соответсвовало закону о товарных знаках в Германии, поэтому KImageShop был переименован в Krayon, который также, по-видимому, ущемлял существующий товарный знак, поэтому Krayon был окончательно переименован в Krita в 2002 году.

Первоначальная разработка была медленной; c 2003 года  она перешла в активную фазу.В  2004 году был выпущен первый публичный релиз с в составле пакета KOffice 1.4. В 2005 году Krita получила поддержку цветовых моделей CMYK, Lab, YCbCr, XYZ и каналов с высокой битовой глубиной, а также поддержку OpenGL.

С 2004 по 2009 год разработка была сфокусирована на повторение функций Photoshop/Gimp. С 2009 года  акцент в разработке был смещен на удоволетворение потребности художников. Сообщество Krita стремится сделать Krita лучшим приложением для рисования для карикатуристов, иллюстраторов и художников-концептов\cite{book2}
\subsection{Функциональные возможности}
Krita поддерживает работу в различных цветовых пространствах и с различными цветовыми моделями — RGB, CMYK, Lab, в режиме от восьми до 32 с плавающей точкой разрядов на канал. Кроме того, реализованы популярные фильтры (такие как нерезкое маскирование), корректирующие слои, маски и динамические фильтры, а также серия инструментов для ретуши.

Однако основным приоритетом разработчики ставят реализацию возможностей для художников. Для них Krita может предложить:
\begin{itemize}
 \item полноценные инструменты для работы с покадровой анимацией, включая экспорт анимации с использованием ffmpeg
\item широкий выбор кистей (в том числе смешивающие, фильтрующие, эффектные, спрей, кисти для заполнения объемов)
\item большое количество режимов наложения
\item управление динамикой кистей с помощью графического планшета
\item имитацию бумаги и пастели
\item поворот и зеркалирование холста
\item псевдо-бесконечный холст
\item поддержку горячих клавиш Photoshop
\end{itemize}












%%% Local Variables:
%%% mode: latex
%%% TeX-master: "rpz"
%%% End:
